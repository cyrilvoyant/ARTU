\section{Experimental setup}
\label{sec:experiment}
To compare the SRMs presented in this paper as objectively as possible, we must establish rules follow best practices in the fields of meteorology, renewable energy, and particularly, solar irradiance forecasting. The seasonality of considered time series and ad-hoc test (based on auto-correlation) are discussed in \ref{sec:season}. In what follows, $I_{GH}$ and $I_{CS}$ are used to denote GHI and its clear-sky expectation, whereas $\widehat{I}_{GH}$ is the GHI forecast. To be more precise, the clear-sky index, which is defined as $\kappa=I_{GH}/I_{CS}$, is used as $x$ as in Eq.~(\ref{eq:eq1}).

\subsection{The pre-treatment}
\label{sec:pre}
As part of this study, several rules and explanations must be given to improve the objectivity of the solar energy forecasting conclusions:
\begin{itemize}[label=$\looparrowright$]
\item Irradiance time series ($I_{GH}$) are measured in several sites around the world with different climatic characteristics. For each of them, the K\"oppen--Geiger classification ($KG$) \citep{ASCENCIOVASQUEZ2019672}, the expected forecastability ($F$) \citep {doi:10.1063/5.0042710} and the geographic coordinates are provided. At least two years of data are available for each site and used during the simulations.
\item  The models are evaluated during daytime irradiance values only, filtering the checked data (according to quality control \citep{espinar_quality_2012,articleCQ}) of the solar zenith angle ($I_{GH}$ where zenith angle $\theta_Z>85^\circ$ are excluded),
\item The data was acquired every 15 min or every hour, and the corresponding forecast horizons are between 15 and 150 min, in the first case, and between 1 and 10 h, in the second case. 
\end{itemize}

\subsection{Error Metrics}
In order to compare the accuracy of the forecasting methods, we refer to the official accuracy measures of the M4 competition, i.e., the mean absolute scaled error (MASE \citep{HYNDMAN2006679}) in Eq.(\ref{eq:26})  computed for seasonal time series in retrospective case. MASE admits an average value as denominator, so as long as the filtering of the night hours is operated (see previous subsection), the denominator in Eq.(\ref{eq:26}) will be neither equal nor close to 0. This metric should not be used alone, it is only computed to improve and validate results obtained with the usual methods (Eqs.(\ref{eq:nmae}-\ref{eq:nrmse}) \citep{YANG202020}). Error is calculated on the out-sample data and averaged over all horizons, the main interest of MASE as defined in Eq.(\ref{eq:26}) is tied with that one coefficient for all horizons (retrospective case with multiple step ahead forecasts). 

\begin{equation}
\label{eq:26}
    \textnormal{MASE}\simeq \frac{100}{h}\frac{\sum_{t\in \textnormal{Test}}\sum_{i=1}^{h}\lvert I_{GH}(t+i)-\widehat{I}_{GH}(t+i)\lvert}{\sum_{t\in \textnormal{Test}}\lvert I_{GH}(t)-I_{GH}(t-m)\lvert},
\end{equation}
where $m$ is the period, see Eq.~(\ref{eq:24},and $\textnormal{Test}$ indicates the test sample of size $n$ with $n\gg m$. The MASE method requires a normalization, performed here using the signal period. The filtration presented previously (Section \ref{sec:pre}) has an adverse effect where it reduces the number of exploitable data but also modifies the seasonality (not constant during the year). As a result, it is possible that the denominator of Eq.~(\ref{eq:26}) takes quite high values, which may result in quite low MASE. As all methods are evaluated identically, the interpretation of MASE remains valid, besides a non-periodic version could have been used.

If these metrics are the most commonly used by researchers working on time series forecasting, the analysis of the literature shows that they are only rarely (if ever) used to validate predictions related to global radiation or photo-voltaic power. There are simpler ones that are used in meteorology and more particularly in deterministic solar resource forecasting \citep{YANG202020}, i.e., the normalized mean absolute error (nMAE), see Eq.~(\ref{eq:nmae}) and the normalized root mean square error (nRMSE), see Eq.~(\ref{eq:nrmse}) \citep{metrics}.

\begin{equation}
\label{eq:nmae}
    \textnormal{nMAE}(h)=100\frac{\sum_{t \in \textnormal{Test}}\lvert I_{GH}(t+h)-\widehat{I}_{GH}(t+h)\lvert}{\sum_{t \in \textnormal{Test}} I_{GH}(t)},
\end{equation}

\begin{equation}
\label{eq:nrmse}
    \textnormal{nRMSE}(h)=100\sqrt{n}\frac{\sqrt{\sum_{t \in \textnormal{Test}}[ I_{GH}(t+h)-\widehat{I}_{GH}(t+h)]^2}}{\sum_{t \in \textnormal{Test}} I_{GH}(t)}.
\end{equation}
Keeping in mind that ARTU was built from the minimization of the $L^2$ norm error function (MSE), it would seem logical in the following that the contribution of the filtering is more consistent with nRMSE than with nMAE but it is imperative to quantify what is happening for both.